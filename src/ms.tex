%%%%%%%%%%%%%%%%%%%%%%%%%%%%%%%%%%%%%%%%%%%%%%%%%%%%%%%%%%%%%%%%%%%%%%%%%%%%%%%
%
%   Please read the CONTRIBUTING.md file in this repository for notes about
%   style and latex conventions!
%
%%%%%%%%%%%%%%%%%%%%%%%%%%%%%%%%%%%%%%%%%%%%%%%%%%%%%%%%%%%%%%%%%%%%%%%%%%%%%%%

\documentclass[modern]{aastex631}
\usepackage{xspace}
\usepackage[utf8]{inputenc}
\usepackage[T1]{fontenc}
\usepackage{ulem}

% To allow putting figures in a subdir
\graphicspath{{figures/}}

\submitjournal{ApJ}

\shorttitle{Astropy Project III}
\shortauthors{The Astropy Collaboration}

\newcommand{\escapecmd}[1]{\texttt{\detokenize{#1}}}

% Packages / projects / programming - for consistency!
\newcommand{\package}[1]{\texttt{#1}\xspace}
\newcommand{\github}{\package{GitHub}}
\newcommand{\python}{\package{Python}}
\newcommand{\astropy}{Astropy\xspace}
\newcommand{\astropypkg}{\package{astropy}}
\newcommand{\mission}[1]{\textit{#1}\xspace}

% For consistency:
\newcommand{\sectionname}{Section\xspace}
\renewcommand{\figurename}{Figure\xspace}
\newcommand{\equationname}{Equation\xspace}
\renewcommand{\tablename}{Table\xspace}

% Words that should not be hyphenated
\hyphenation{NumFOCUS}

% For commenting - can be deleted before submission
\usepackage[colorinlistoftodos]{todonotes}
\newcommand{\inlinecomment}[2]{\todo[inline]{#1: #2}\xspace}
\newcommand{\comment}[2]{\todo{#1: #2}\xspace}

\usepackage{newunicodechar,graphicx}
\DeclareRobustCommand{\okina}{%
 \raisebox{\dimexpr\fontcharht\font`A-\height}{%
 \scalebox{0.8}{`}%
 }%
}
\newunicodechar{ʻ}{\okina}

% Links to documentation: subpackages
\newcommand{\astropysubpkg}[1]{\href{http://docs.astropy.org/en/stable/#1/index.html}{\texttt{astropy.#1}}\xspace}
\newcommand{\astropyiosubpkg}[1]{\href{http://docs.astropy.org/en/stable/io/#1/index.html}{\texttt{astropy.io.#1}}\xspace}
\newcommand{\astropywcsaxes}{\href{http://docs.astropy.org/en/stable/visualization/wcsaxes/index.html}{\texttt{astropy.visualization.wcsaxes}}\xspace}
\newcommand{\astropycosmology}{\astropysubpkg{cosmology}}
\newcommand{\astropycosmologyunits}{\href{https://docs.astropy.org/en/stable/cosmology/units.html}{\texttt{astropy.cosmology.units}}}
\newcommand{\astropyunits}{\astropysubpkg{units}}
\newcommand{\astropycoordinates}{\astropysubpkg{coordinates}}
\newcommand{\astropyconstants}{\astropysubpkg{constants}}
\newcommand{\astropytable}{\astropysubpkg{table}}
\newcommand{\astropytime}{\astropysubpkg{time}}
\newcommand{\astropymodeling}{\astropysubpkg{modeling}}
\newcommand{\astropywcs}{\astropysubpkg{wcs}}
\newcommand{\astropyfits}{\astropyiosubpkg{fits}}

% Links to documentation: classes
\newcommand{\astropyapi}[2]{\href{https://docs.astropy.org/en/stable/api/astropy.#1.html}{#2}}
\newcommand{\astropyapidoc}[2]{\astropyapi{#1}{\texttt{#2}\xspace}}

\newcommand{\astropySpectralCoord}{\astropyapidoc{coordinates.SpectralCoord}{SpectralCoord}}
\newcommand{\astropySkyCoord}{\astropyapidoc{coordinates.SkyCoord}{SkyCoord}}
\newcommand{\astropyICRS}{\astropyapidoc{coordinates.builtin_frames.ICRS}{ICRS}}
\newcommand{\astropyGalacticLSR}{\astropyapidoc{coordinates.builtin_frames.
GalacticLSR}{GalacticLSR}}
\newcommand{\astropyAltAz}{\astropyapidoc{coordinates.builtin_frames.AltAz}{AltAz}}
\newcommand{\astropyGalactocentric}{\astropyapidoc{coordinates.builtin_frames.Galactocentric}{Galactocentric}}
\newcommand{\astropyCosmology}{\astropyapidoc{cosmology.Cosmology}{Cosmology}}
\newcommand{\astropyFlatLambdaCDM}{\astropyapidoc{cosmology.FlatLambdaCDM}{FlatLambdaCDM}}
\newcommand{\astropyFitsColumn}{\astropyapidoc{io.fits.Column}{Column}}
\newcommand{\astropyModel}{\astropyapidoc{modeling.Model}{Model}}
\newcommand{\astropyTableColumn}{\astropyapidoc{table.Column}{Column}}
\newcommand{\astropyTable}{\astropyapidoc{table.Table}{Table}}
\newcommand{\astropyQTable}{\astropyapidoc{table.QTable}{QTable}}
\newcommand{\astropyTime}{\astropyapidoc{time.Time}{Time}}
\newcommand{\astropyLeapSeconds}{\astropyapidoc{time.LeapSeconds}{LeapSeconds}}
\newcommand{\astropyDistribution}{\astropyapidoc{uncertainty.Distribution}{Distribution}}
\newcommand{\astropyUnit}{\astropyapidoc{units.Unit}{Unit}}
\newcommand{\astropyQuantity}{\astropyapidoc{units.Quantity}{Quantity}}
\newcommand{\astropyMasked}{\astropyapidoc{utils.masked.Masked}{Masked}}
\newcommand{\astropyScienceState}{\astropyapidoc{utils.state.ScienceState}{ScienceState}}

% suggestion: shortcut reference to Paper II
\newcommand{\paperii}{\cite{astropy:2018}}

% Maybe have command link to Zenodo reference instead, or include it?
\newcommand{\astropyAPE}[1]{\href{https://github.com/astropy/astropy-APEs/blob/main/APE#1.rst}{APE #1}\xspace}

% Links to documentation: subpackages
\newcommand{\astropysubpkg}[1]{\href{http://docs.astropy.org/en/stable/#1/index.html}{\texttt{astropy.#1}}\xspace}
\newcommand{\astropyiosubpkg}[1]{\href{http://docs.astropy.org/en/stable/io/#1/index.html}{\texttt{astropy.io.#1}}\xspace}
\newcommand{\astropywcsaxes}{\href{http://docs.astropy.org/en/stable/visualization/wcsaxes/index.html}{\texttt{astropy.visualization.wcsaxes}}\xspace}

% Links to documentation: classes
\newcommand{\astropyskycoord}{\href{http://docs.astropy.org/en/stable/api/astropy.coordinates.SkyCoord.html}{\texttt{SkyCoord}}\xspace}
\newcommand{\astropyQuantity}{\href{http://docs.astropy.org/en/stable/api/astropy.units.Quantity.html}{\texttt{Quantity}}\xspace}
\newcommand{\astropyTime}{\href{http://docs.astropy.org/en/stable/api/astropy.time.Time.html}{\texttt{Time}}\xspace}
\newcommand{\astropyTable}{\href{http://docs.astropy.org/en/stable/api/astropy.table.Table.html}{\texttt{Table}}\xspace}

\begin{document}

\draft{\today}

\title{The Astropy Project: xx}

\correspondingauthor{Astropy Coordination Committee}
\email{coordinators@astropy.org}

\author{Astropy Collaboration}
\noaffiliation
{\let\thefootnote\relax\footnote{{The author list has two parts: the authors that made significant contributions to the writing of the paper in order of contribution, followed by contributors to the \astropy Project in alphabetical order. \textbf{The position in the author list does not correspond to contributions to the \astropy Project as a whole.} A more complete list of contributors to the core package can be found in the \href{https://github.com/astropy/astropy/graphs/contributors}{package repository}, and at the \href{http://www.astropy.org/team.html}{\astropy team webpage}.}}}

% % \newcommand{\afprinceton}{Department of Astrophysical Sciences, Princeton University, Princeton, NJ 08544, USA}
% \newcommand{\afstsci}{Space Telescope Science Institute, 3700 San Martin Dr., Baltimore, MD 21218, USA}
% \newcommand{\afsaao}{South African Astronomical Observatory, PO Box 9, Observatory 7935, Cape Town, South Africa}
% \newcommand{\afminnstate}{Department of Physics and Astronomy, Minnesota State University Moorhead, 1104 7th Ave S, Moorhead, MN 56563}
% \newcommand{\afgoddard}{NASA Goddard Space Flight Center, 8800 Greenbelt Road, Greenbelt, MD 20771, USA}
% \newcommand{\afcfa}{Harvard-Smithsonian Center for Astrophysics, 60 Garden St., Cambridge, MA, 02138, USA}
% \newcommand{\afwesternontario}{Department of Physics \& Astronomy, University of Western Ontario, 1151 Richmond St, London ON N5X4H1 Canada}
% \newcommand{\afnasaames}{NASA Ames Research Center, Moffett Field, CA 94043, USA}
% \newcommand{\afjhu}{Department of Physics and Astronomy, Johns Hopkins University, Baltimore, MD 21218, USA}
% \newcommand{\afmpik}{Max-Planck-Institut f\"ur Kernphysik, PO Box 103980, 69029 Heidelberg, Germany}
% \newcommand{\afioa}{Institute of Astronomy, University of Cambridge, Madingley Road, Cambridge, CB3 0HA, UK}
% \newcommand{\afpennstate}{Dept of Astronomy and Astrophysics, Pennsylvania State University, University Park, PA 16802}
% \newcommand{\afgeminin}{Gemini Observatory, 670 N. Aohoku Pl, Hilo, HI 96720, USA}
% \newcommand{\afgeminis}{Gemini Observatory, Casilla 603, La Serena, Chile}
% \newcommand{\aflco}{Las Cumbres Observatory, 6740 Cortona Drive, Suite 102, Goleta, CA 93117-5575, USA}
% \newcommand{\afucsb}{Department of Physics, University of California, Santa Barbara, CA 93106-9530, USA}
% \newcommand{\afesomunich}{European Southern Observatory, Karl-Schwarzschild-Stra{\ss}e 2, 85748 Garching bei M\"{u}nchen, Germany}
% \newcommand{\afuw}{Department of Astronomy, University of Washington, Seattle, WA 98155}
% \newcommand{\afberkeleyastro}{Department of Astronomy, UC Berkeley, 501 Campbell Hall \#3411, Berkeley, CA 94720, USA}
% \newcommand{\afuct}{Department of Astronomy, University of Cape Town, Private Bag X3, Rondebosch 7701, South Africa}

% \author[0000-0003-0872-7098]{A. M. Price-Whelan}
% \affiliation{\afprinceton}

% \author[0000-0002-3713-6337]{B. M. Sip\H{o}cz}
% \noaffiliation

% \author[0000-0003-4243-2840]{H. M. G\"{u}nther}
% \affiliation{Kavli Institute for Astrophysics and Space Research, Massachusetts Institute of Technology, 70 Vassar St., Cambridge, MA 02139, USA}

% \author[0000-0003-0079-4114]{P. L. Lim}
% \affiliation{\afstsci}

% \author[0000-0002-8969-5229]{S. M. Crawford}
% \affiliation{\afsaao}

% \author[0000-0002-3657-4191]{S. Conseil}
% \affiliation{Univ Lyon, Univ Lyon1, Ens de Lyon, CNRS, Centre de Recherche Astrophysique de Lyon UMR5574, F-69230, Saint-Genis-Laval, France}

% \author[0000-0003-4401-0430]{D. L. Shupe}
% \affiliation{Caltech/IPAC, 1200 E. California Blvd, Pasadena, CA 91125}

% \author[0000-0001-7988-8919]{M. W. Craig}
% \affiliation{\afminnstate}

% \author[0000-0002-5686-9632]{N. Dencheva}
% \affiliation{\afstsci}

% \author[0000-0001-6431-9633]{A. Ginsburg}
% \affiliation{National Radio Astronomy Observatory, 1003 Lopezville Rd, Socorro, NM 87801}

% \author[0000-0002-9623-3401]{J. T. VanderPlas}
% \affiliation{eScience Institute, University of Washington, 3910 15th Ave NE, Seattle, WA 98195, USA}

% \author[0000-0002-7908-9284]{L. D. Bradley}
% \affiliation{\afstsci}

% \author[0000-0003-0784-6909]{D. P\'{e}rez-Su\'{a}rez}
% \affiliation{University College London/Research IT Services, Gower St, Bloomsbury, London WC1E 6BT, United Kingdom}

% \author[0000-0002-0455-9384]{M. de Val-Borro}
% \affiliation{Astrochemistry Laboratory, \afgoddard}

% \collaboration{(primary paper contributors)}

% \author{T. L. Aldcroft}
% \affiliation{\afcfa}

% \author[0000-0002-1821-0650]{K. L. Cruz}
% \affiliation{Department of Physics and Astronomy, Hunter College, City University of New York, 695 Park Avenue, New York, NY 10065}
% \affiliation{Physics, Graduate Center of the City University of New York, New York, NY, USA}
% \affiliation{Department of Astrophysics, American Museum of Natural History, New York, NY, USA}
% \affiliation{Center for Computational Astropyhsics, Flatiron Institute, 162 Fifth Avenue, New York, NY 10010, USA}

% \author[0000-0002-8642-1329]{T. P. Robitaille}
% \affiliation{Aperio Software Ltd., Headingley Enterprise and Arts Centre, Bennett Road, Leeds, LS6 3HN, United Kingdom}

% \author[0000-0002-9599-310X]{E. J. Tollerud}
% \affiliation{\afstsci}

% \collaboration{(Astropy coordination committee)}

% \author{C. Ardelean}
% \affiliation{\afwesternontario}

% \author[0000-0002-8222-3595]{T. Babej}
% \affiliation{Department of Theoretical Physics \& Astrophysics, Masaryk University, Kotlarska 2, 61137 Brno, Czech Republic}

% \author[0000-0002-2618-1124]{Y. P. Bach}
% \affiliation{Department of Physics and Astronomy, Seoul National University, Gwanak-gu, Seoul 08826, South Korea}

% \author[0000-0002-4576-9337]{M. Bachetti}
% \affiliation{INAF-Osservatorio Astronomico di Cagliari, via della Scienza 5, I-09047, Selargius, Italy}

% \author{A. V. Bakanov}
% \noaffiliation

% \author[0000-0001-7821-7195]{S. P. Bamford}
% \affiliation{School of Physics \& Astronomy, University of Nottingham, University Park, Nottingham NG7 2RD, UK}

% \author[0000-0002-3306-3484]{G. Barentsen}
% \affiliation{\afnasaames}

% \author[0000-0003-2767-0090]{P. Barmby}
% \affiliation{\afwesternontario}

% \author[0000-0002-9374-2729]{A. Baumbach}
% \affiliation{Heidelberg University, Kirchhoff Institut for Physics, Im Neuenheimer Feld 227, 69116 Heidelberg, Germany}

% \author{K. L. Berry}
% \noaffiliation

% \author{F. Biscani}
% \affiliation{Max-Planck-Institut f\"ur Astronomie, K\"onigstuhl 17, 69117 Heidelberg, Germany}

% \author[0000-0003-0946-6176]{M. Boquien}
% \affiliation{Unidad de Astronomía, Fac. Cs. Básicas, Universidad de Antofagasta, Avda. U. de Antofagasta 02800, Antofagasta, Chile}

% \author{K. A. Bostroem}
% \affiliation{Department of Physics, UC Davis, 1 Shields Ave, Davis, CA, 95616, USA}

% \author{L. G. Bouma}
% \affiliation{\afprinceton}

% \author[0000-0003-2680-005X]{G. B. Brammer}
% \affiliation{\afstsci}

% \author{E. M. Bray}
% \noaffiliation

% \author[0000-0001-5391-2386]{H. Breytenbach}
% \affiliation{\afsaao}
% \affiliation{\afuct}

% \author[0000-0001-8001-0089]{H. Buddelmeijer}
% \affiliation{Leiden Observatory, Leiden University, P.O. Box 9513, 2300 RA, Leiden, The Netherlands}

% \author[0000-0003-4428-7835]{D. J. Burke}
% \affiliation{\afcfa}

% \author[0000-0002-7738-5389]{G. Calderone}
% \affiliation{Istituto Nazionale di Astrofisica, via Tiepolo 11 Trieste, Italy}

% \author[0000-0002-2187-161X]{J. L. Cano Rodríguez}
% \noaffiliation

% \author{M. Cara}
% \affiliation{\afstsci}

% \author{J. V. M. Cardoso}
% \affiliation{Universidade Federal de Campina Grande, Campina Grande, PB 58429-900, Brazil}
% \affiliation{\afnasaames}
% \affiliation{Bay Area Environmental Research Institute, Petaluma, CA 94952, USA}

% \author{S. Cheedella}
% \affiliation{Department of Physics, Virginia Tech, Blacksburg, VA 24061, USA}

% \author[0000-0002-5317-7518]{Y. Copin}
% \affiliation{Universit\'e de Lyon, F-69622, Lyon, France; Universit\'e de Lyon 1, Villeurbanne; CNRS/IN2P3, Institut de Physique Nucl\'eaire de Lyon}

% \author[0000-0002-5466-3817]{L.  Corrales}
% \affiliation{University of Wisconsin - Madison, 475 North Charter Street, Madison, WI 53706}
% \affiliation{Einstein Fellow}

% \author[0000-0003-1204-3035]{D. Crichton}
% \affiliation{\afjhu}

% \author{D. D'Avella}
% \affiliation{\afstsci}

% \author[0000-0002-4198-4005]{C. Deil}
% \affiliation{\afmpik}

% \author[0000-0003-0526-3873]{\'{E}. Depagne}
% \affiliation{\afsaao}

% \author[0000-0002-8134-9591]{J. P. Dietrich}
% \affiliation{Faculty of Physics, Ludwig-Maximilians-Universit\"at, Scheinerstr. 1, 81679 Munich, Germany}
% \affiliation{Excellence Cluster Universe, Boltzmannstr. 2, 85748 Garching b. M\"unchen, Germany}

% \author{A. Donath}
% \affiliation{\afmpik}

% \author{M. Droettboom}
% \affiliation{\afstsci}

% \author[0000-0003-1714-7415]{N. Earl}
% \affiliation{\afstsci}

% \author{T. Erben}
% \affiliation{Argelander-Institut f\"ur Astronomie, Auf dem H\"ugel 71, 53121 Bonn, Germany}

% \author{S. Fabbro}
% \affil{National Research Council Herzberg Astronomy \& Astrophysics, 4071 West Saanich Road, Victoria, BC}

% \author[0000-0002-8919-079X]{L. A. Ferreira}
% \affiliation{Instituto de Matemática Estatística e Física – IMEF, Universidade Federal do Rio Grande – FURG, Rio Grande, RS 96203-900, Brazil}

% \author{T. Finethy}
% \noaffiliation

% \author[0000-0003-4291-1091]{R. T. Fox}
% \noaffiliation

% \author[0000-0002-9853-5673]{L. H. Garrison}
% \affiliation{\afcfa}

% \author{S. L. J. Gibbons}
% \affiliation{\afioa}

% \author{D. A. Goldstein}
% \affiliation{\afberkeleyastro}
% \affiliation{Lawrence Berkeley National Laboratory, 1 Cyclotron Road, Berkeley, CA 94720, USA}

% \author[0000-0002-0300-3333]{R. Gommers}
% \affiliation{Scion, Private Bag 3020, Rotorua, New Zealand}

% \author[0000-0003-4970-2874]{J. P. Greco}
% \affiliation{\afprinceton}

% \author{P. Greenfield}
% \affiliation{\afstsci}

% \author[0000-0002-6508-2938]{A. M. Groener}
% \affiliation{Drexel University, Physics Department, Philadelphia, PA 19104, USA}

% \author{F. Grollier}
% \noaffiliation

% \author[0000-0003-2031-7737]{A. Hagen}
% \affiliation{Vizual.ai, 3600 O'Donnell St, Suite 250, Baltimore, MD 21224}
% \affiliation{\afpennstate}

% \author{P. Hirst}
% \affiliation{\afgeminin}

% \author[0000-0002-8546-9128]{D. Homeier}
% \affiliation{Zentrum f{\"u}r Astronomie der Universit{\"a}t Heidelberg, Landessternwarte, K{\"o}nigstuhl 12, 69117 Heidelberg, Germany}

% \author[0000-0002-4600-7852]{A. J. Horton}
% \affiliation{Australian Astronomical Observatory, 105 Delhi Road, North Ryde NSW 2113, Australia}

% \author[0000-0002-0832-2974]{G. Hosseinzadeh}
% \affiliation{\aflco}
% \affiliation{\afucsb}

% \author{L. Hu}
% \affiliation{Imperial College London,  Kensington, London SW7 2AZ, United Kingdom}

% \author[0000-0003-4989-0289]{J. S. Hunkeler}
% \affiliation{\afstsci}

% \author[0000-0001-5250-2633]{\v{Z}. Ivezi\'{c}}
% \affiliation{\afuw}

% \author{A. Jain}
% \affiliation{BITS PILANI/Computer Science, Pilani Campus, Rajasthan, India}

% \author[0000-0001-5982-167X]{T. Jenness}
% \affiliation{Large Synoptic Survey Telescope, 950 N. Cherry Ave., Tucson, AZ, 85719, USA}

% \author{G. Kanarek}
% \affiliation{\afstsci}

% \author[0000-0002-7612-0469]{S. Kendrew}
% \affiliation{European Space Agency, \afstsci}

% \author[0000-0002-8211-1892]{N. S. Kern}
% \affiliation{\afberkeleyastro}

% \author[0000-0002-0479-7235]{W. E. Kerzendorf}
% \affiliation{\afesomunich}

% \author{A. Khvalko}
% \noaffiliation

% \author{J. King}
% \affiliation{\afmpik}

% \author[0000-0002-8828-5463]{D. Kirkby}
% \affiliation{Department of Physics and Astronomy, University of California, Irvine, CA 92697, USA}

% \author{A. M. Kulkarni}
% \affiliation{College of Engineering Pune/Department of Computer Engineering and IT, Shivajinagar, Pune 411005, India}

% \author{A. Kumar}
% \affiliation{Delhi Technological University}

% \author[0000-0003-2193-5369]{A. Lee}
% \affiliation{Department of Physics, University of Berkeley, Califonia, CA94709, USA}

% \author[0000-0001-5820-475X]{D. Lenz}
% \affiliation{Jet Propulsion Laboratory, California Institute of Technology, 4800 Oak Grove Drive, Pasadena, CA 91109, USA}

% \author[0000-0001-7221-855X]{S. P. Littlefair}
% \affiliation{Department of Physics \& Astronomy, University of Sheffield, Sheffield, S3 7RH, UK}

% \author[0000-0003-3270-6844]{Z. Ma}
% \affiliation{Department of Physics and Astronomy, University of Missouri, Columbia, Missouri, 65211, USA}

% \author[0000-0002-1395-8694]{D. M. Macleod}
% \affiliation{Cardiff University, Cardiff CF24 3AA, UK}

% \author[0000-0002-6324-5713]{M. Mastropietro}
% \affiliation{Department of Physics and Astronomy, Ghent University, Krijgslaan 281, S9, B-9000 Gent, Belgium}

% \author[0000-0001-5807-7893]{C. McCully}
% \affiliation{\aflco}
% \affiliation{\afucsb}

% \author{S. Montagnac}
% \affiliation{Puy-Sainte-R\'eparade Observatory}

% \author[0000-0003-2528-3409]{B. M. Morris}
% \affiliation{\afuw}

% \author{M. Mueller}
% \affil{Department of Mathematics, Brown University, 151 Thayer Street, Providence, RI 02912, USA}

% \author[0000-0003-4217-4642]{S. J. Mumford}
% \affiliation{SP$^{2}$RC, School of Mathematics and Statistics, The University of Sheffield, U.K.}

% \author[0000-0002-1631-4114]{D. Muna}
% \affiliation{Center for Cosmology and Astroparticle Physics, The Ohio State University, 191 West Woodruff Avenue, Columbus, OH 43210}

% \author[0000-0001-6628-8033]{N. A. Murphy}
% \affiliation{\afcfa}

% \author{S. Nelson}
% \affiliation{\afminnstate}

% \author[0000-0002-1966-3627]{G. H. Nguyen}
% \affiliation{VNU-HCMC, University of Natural Sciences/Faculty of IT, 227 Nguyen Van Cu St., Ward 4, District 5, Ho Chi Minh City, Vietnam}

% \author[0000-0001-8720-5612]{J. P. Ninan}
% \affiliation{\afpennstate}

% \author{M. N{\"o}the}
% \affiliation{Experimental Physics 5, TU Dortmund, Otto-Hahn-Str. 4, 44227 Dortmund, Germany}

% \author{S. Ogaz}
% \affiliation{\afstsci}

% \author[0000-0001-7790-5308]{S. Oh}
% \affiliation{\afprinceton}

% \author{J. K. Parejko}
% \affiliation{\afuw}

% \author{N. Parley}
% \affiliation{University of Reading, Whiteknights Campus, Reading RG6 6BX, UK}

% \author[0000-0002-9351-6051]{S. Pascual}
% \affiliation{Departamento de Astrofisica, Universidad Complutense de Madrid, Madrid, Spain}

% \author{R. Patil}
% \noaffiliation

% \author{A. A. Patil}
% \affiliation{Pune Institute of Computer Technology, Pune 411043, India}

% \author[0000-0002-9912-5705]{A. L. Plunkett}
% \affiliation{European Southern Observatory, Av. Alonso de C\'{o}rdova 3107, Vitacura, Santiago, Chile}

% \author{J. X. Prochaska}
% \affiliation{Astronomy \& Astrophysics, UC Santa Cruz, 1156 High St., Santa Cruz, CA 95064 USA}

% \author{T. Rastogi}
% \noaffiliation

% \author{V. Reddy Janga}
% \affiliation{Indian Institute of Technology, Mechanical Engineering, Kharagpur, India}

% \author[0000-0003-1149-6294]{J. Sabater}
% \affiliation{Institute for Astronomy (IfA), University of Edinburgh, Royal Observatory, Blackford Hill, EH9 3HJ Edinburgh, U.K.}

% \author{P. Sakurikar}
% \affiliation{IIIT-Hyderabad, India}

% \author{M. Seifert}
% \noaffiliation

% \author{L. E. Sherbert}
% \affiliation{\afstsci}

% \author[0000-0003-0477-6220]{H. Sherwood-Taylor}
% \noaffiliation

% \author{A. Y. Shih}
% \affiliation{\afgoddard}

% \author[0000-0003-3001-676X]{J. Sick}
% \affil{AURA/LSST, 950 N Cherry Ave, Tucson, 85719}

% \author{M. T. Silbiger}
% \noaffiliation

% \author[0000-0003-2462-7273]{S. Singanamalla}
% \affiliation{Microsoft Research}

% \author[0000-0001-9898-5597]{L. P. Singer}
% \affiliation{Astroparticle Physics Laboratory, \afgoddard}
% \affiliation{Joint Space-Science Institute, University of Maryland, College Park, MD 20742, USA}

% \author[0000-0003-1585-225X]{P. H. Sladen}
% \affiliation{Zentrum f{\"u}r Astronomie der Universit{\"a}t Heidelberg, Astronomisches Rechen-Institut, M{\"o}nchhofstra{\ss}e 12--14, 69120 Heidelberg, Germany}

% \author{K. A. Sooley}
% \noaffiliation

% \author{S. Sornarajah}
% \noaffiliation

% \author[0000-0001-7751-1843]{O. Streicher}
% \affiliation{Leibniz Institute for Astrophysics Potsdam (AIP), An der Sternwarte 16, 14482 Potsdam, Germany}

% \author[0000-0003-1774-3436]{P. Teuben}
% \affiliation{Astronomy Department, University of Maryland, College Park, MD. 20742, USA}

% \author{S. W. Thomas}
% \affiliation{\afioa}

% \author[0000-0002-5445-5401]{G. R. Tremblay}
% \affiliation{{\afcfa}}


% \author{J. E. H. Turner}
% \affiliation{\afgeminis}

% \author{V. Terr\'{o}n}
% \affiliation{Institute of Astrophysics of Andalusia (IAA-CSIC), Granada, Spain}

% \author[0000-0002-5830-8505]{M. H. van Kerkwijk}
% \affiliation{Department of Astronomy \& Astrophysics, University of Toronto, 50 Saint George Street, Toronto, ON M5S 3H4, Canada}

% \author[0000-0002-6219-5558]{A. de la Vega}
% \affiliation{\afjhu}

% \author[0000-0002-1343-134X]{L. L. Watkins}
% \affiliation{\afstsci}

% \author{B. A. Weaver}
% \affiliation{National Optical Astronomy Observatory, 950 N. Cherry Ave., Tucson, AZ 85719, USA}

% \author[0000-0003-4824-2087]{J. B. Whitmore}
% \affiliation{Centre for Astrophysics and Supercomputing, Swinburne University of Technology, Hawthorn, VIC 3122, Australia}

% \author[0000-0002-2958-4738]{J. Woillez}
% \affiliation{\afesomunich}


% \author[0000-0003-2638-7648]{V. Zabalza}
% \noaffiliation

% \collaboration{(Astropy contributors)}

\author{Astropy Contributor 1}
\noaffiliation


\begin{abstract}

    To be written...

\end{abstract}

\keywords{%
    Astrophysics - Instrumentation and Methods for Astrophysics
    ---
    methods: data analysis
    ---
    methods: miscellaneous
}


\section{Introduction} \label{sec:intro}

The \python programming language is a high-level, interpreted (versus compiled)
programming language that has become an industry standard across many
computational domains, technological sectors, and fields of research.
Despite claims to the contrary \citep{Portegies-Zwart:2020}, \python enables
scalable, time- and energy-efficient code execution \citep[e.g.,][]{Augier:2021}
with a focus on code readability, ease of use, and interoperability with other
languages.
Over the last decade, \python has grown enormously in popularity to become a
dominant programming language in the astronomical and broader scientific
communities.
For example, Figure~\ref{fig:python-mentions} shows the number of yearly
full-text mentions of \python as compared to a few other programming languages
in refereed articles in the astronomical literature, demonstrating its nearly
exponential growth in popularity.
The rapid adoption of \python by astronomy researchers, students, observatories,
and technical staff combined with an associated increase in awareness and
interest about open-source software tools is contributing to a paradigm shift in
the way research is done, data is analyzed, and results are shared in astronomy
and beyond.

One of the factors that has led to its rapid ascent in popularity in scientific
contexts has been significant volunteer-driven effort behind developing
community-oriented open-source software tools and fostering communities of users
and developers that have grown around these packages.
TODO: Numpy, scipy, matplotlib, scientific software ecosystem.
The astropy core package ... one of the first large, community-oriented, open-source \python packages ... was Astropy ... (note: yt)
TODO: words about astropy core package.
TODO: The core package is now largely stable and provides core foundational tools.

TODO: Astropy has expanded to become the Astropy Project: Broader scope than core package. Core package, Ecosystem of tools, Community of users, developers, maintainers

TODO:  https://physicsworld.com/a/standing-on-the-shoulders-of-programmers/

Transition of needs: From new features, to community development, sustainability, ...
- pipeline between User, Participant, Code contributor, Maintainer, Coordinator.

Funding received:
- Moore
- NASA




\begin{figure}
    \begin{centering}
        \includegraphics[width=\textwidth]{figures/python-mentions.pdf}
        \caption{
            Yearly full-text mentions of programming languages (indicated in the
            figure legend) in refereed publications in the astronomical
            literature database in the Astrophysics Data System (ADS;
            \citealt{ads}).
            \python has rapidly become the dominant programming language
            mentioned in refereed articles over the last 10 years.
        }
        \label{fig:python-mentions}
    \end{centering}
\end{figure}


\section{Major Updates in the Astropy Project} \label{sec:project-updates}


\subsection{Project governance} \label{sec:project-governance}

New CoCo and election overview.


\subsection{Contributor base} \label{sec:project-contributors}

Overview and statistics of contributors. Changes since v2.0.

\begin{figure}
    \begin{centering}
        \includegraphics{figures/contributor-summary.pdf}
        \caption{Placeholder figure!}
        \label{fig:contributor-summary}
    \end{centering}
\end{figure}



\subsection{Funding} \label{sec:project-funding}

New funding sources.


\section{Major Updates to the Astropy Core Package} \label{sec:core-updates}

\subsection{New Long-term Support (LTS) Version: v5.0} \label{sec:core-v50}

\subsection{Highlighted Feature Development} \label{sec:core-features}

\begin{itemize}
    \item Support for representing and transforming velocity data in coordinates, epoch propagation (v3.0)
    \item Improved support for astronomical time series: TimeSeries object (v3.2), Box Least Squares periodogram (v3.1)
    \item Overall improved support of Quantity throughout numpy (v4.0) and scipy
    \item Native support for Time, Quantity, and SkyCoord objects in Astropy tables
    \item TODO: something about WCS, SpectralCoord?
\end{itemize}


\section{Learn Astropy} \label{sec:learn}

Current status and scope

Vision for the future (review what we said in v2.0 paper)

User forums and engagement with user base


\section{Supporting the Ecosystem of Astronomical Python Software}
\label{sec:ecosystem}

\subsection{Community-oriented infrastructure}

Infrastructure packages that exist to support and help others maintain their package infrastructure.

\subsection{Affiliated packages}

New packages and major updates!

\subsection{Connections with observatories}

JWST (motivated by excitement around the launch?)

Gemini


\section{Future Plans for the Astropy Project} \label{sec:future}

\subsection{Roadmap}

\subsection{Challenges}

Attracting new contributors when the code has become quite complex,
Contributor to maintainer mentoring,
Long-term / sustained funding for maintaining infrastructure,
...


\begin{acknowledgments}

We would like to thank the members of the community that have contributed to
\astropy, that have opened issues and provided feedback, and have supported the
project in a number of different ways.

The \astropy community is supported by and makes use
of a number of organizations and services outside the traditional
academic community. We thank Google for financing and organizing the
Google Summer of Code (GSoC) program, that has funded severals
students per year to work on \astropy related projects over the
summer. These students often turn into long-term contributors. We also
thank NumFOCUS and the Python Software Foundation for financial
support. Within the academic community, we thank
institutions that make it possible that astronomers and other developers on
their staff can contribute their time to the development of
\astropy projects.  We would like acknowledge the support of the
Space Telescope Science Institute, Harvard–Smithsonian Center for Astrophysics,
and the South African Astronomical Observatory.

Furthermore, the \astropy packages would not exist in their current form without
a number of web services for code hosting, continuous integration, and
documentation; in particular, \astropy heavily relies on GitHub, Travis CI,
Appveyor, CircleCI, and Read the Docs.

\astropypkg interfaces with the SIMBAD database, operated at CDS, Strasbourg,
France. It also makes use of the ERFA library \citep{erfa}, which in turn
derives from the IAU SOFA Collection\footnote{\url{http://www.iausofa.org}}
developed by the International Astronomical Union Standards of Fundamental
Astronomy \citep{sofa}.

\end{acknowledgments}

\software{\package{astropy} (\citealt{astropy}),
          \package{numpy} (\citealt{numpy}),
          \package{scipy} (\citealt{scipy}),
          \package{matplotlib} (\citealt{matplotlib}),
          \package{Cython} (\citealt{cython}),
          \package{SOFA} (\citealt{sofa}),
          \package{ERFA} (\citealt{erfa})
          }

\bibliographystyle{aasjournal}
\bibliography{bibliography}


% \appendix

% \section{List of Affiliated Packages}

% \begin{longrotatetable}
%     \begin{deluxetable*}{cccp{3in}c}
%     \tablecaption{Registry of affiliated packages.}
%     \label{tab:registry}
%     \tablehead{
%         \colhead{Package Name} &
%         \colhead{Stable} &
%         \colhead{PyPI Name} &
%         \colhead{Maintainer} &
%         \colhead{Citation}
%       }
%       \startdata
%         \input{registry.tex}
%       \enddata
%   \end{deluxetable*}
% \end{longrotatetable}


\end{document}
